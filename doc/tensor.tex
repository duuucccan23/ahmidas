\documentclass[a4paper,10pt]{article}
\usepackage[utf8x]{inputenc}

%opening
\title{Tensor calculus and data structures}
\author{Ahmidas dev team}

\begin{document}

\section{Introduction}
The most mathematically complex part of Ahmidas involves contractions, where objects such as propagators have both spacetime indices and gauge (colour) indices. Because keeping track of these indices can be a programmatical nightmare, we try to clearly separate them in the code. The way this is done is by using the Field construction. A field is a well defined C++ object that has a certain element at every lattice site. This element is the first part of a choice, and we have chosen to have the gauge (colour) indices all contained at a single lattice site. Dirac or spacetime indices are external to the Field, such that in essence to describe fully a Gauge Field, one needs 4 Fields of $SU(3)$ matrices. And more precisely, to fully describe a Propagator with two spacetime indices and 2 colour indices, one needs 4x4=16 Fields each having a 3x3 2-Tensor of colour at every lattice site. The aim of this document is to describe the mathematical properties of these objects, and to link those to the used data structures.
\section{Mathematical objects}
$SU(N)$ Matrix: is a N by N complex valued matrix, to be distinguished from a 2-Tensor.
An m-Tensor has both a rank ($m$) and a dimension ($N$). It has $m^{N}$ components.
A 1-Tensor is also known as a Vector.
Scalar: complex number, this is NOT called a 0-Tensor for reasons that will become clear.

\section{Mathematical operations}
\subsection{Contraction}
\section{Data structures}

\section{Functions}

\end{document}
